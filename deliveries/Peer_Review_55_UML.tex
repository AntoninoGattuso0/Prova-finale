\documentclass[12pt]{article}
\usepackage[utf8]{inputenc}
\usepackage[T1]{fontenc}
\usepackage[italian]{babel}

\title{Peer-Review 55: UML}
\author{Gambarin Paolo, Gattuso Antonino, Mersam Rebeca\\Gruppo 56}

\begin{document}

\maketitle

Valutazione del diagramma UML delle classi del gruppo 55.

\section{Lati positivi}

Un lato positivo che abbiamo trovato è quello dell'utilizzo di un array di interi per contare il numero di pedine per ogni colore.

\section{Lati negativi}

Purtroppo in questo UML abbiamo trovato non pochi lati negativi:

        -i nomi utilizzati per rappresentare le variabili e i metodi sono poco chiari e molto spesso vengono utilizzati per rappresentare variabili di classe diverse (es. id)
        -alcuni attributi cambiano tipo in classi diverse: prof in "notUsedPieces" viene indicato come booleano, mentre in PlayerBoard viene indicato come intero
        -le cardinalità a volte sono sbagliate: ci sono come minimo 2 cloud; ci sono minimo 2 playerboard
        -ci sono alcuni errori per quanto riguarda le variabili in ingresso e uscita dei metodi: in CLOUD "setPlaceForStudent" crediamo debba restituire un intero dato che deve stabilire il numero
            di studenti per ogni nuvola (quindi il nome SET non è il più appropriato); "getStudentCloud" dovrebbe avere in ingresso l'ID della nuvola e restituire un array di interi contenente
            il numero di studenti per ogni colore presente sulla nuvola
        -per quanto riguarda la classe ISLAND il numero di torri non è necessario dato che possono essere presenti 0 oppure 1 torre. Inoltre la dominanza non è determinata da un singolo colore, ma da tutte
            le pedine dello stesso colore dei professori appartenenti a un determinato player
        -la classe che indica i pezzi non usati è molto confusionaria dal punto di vista logico e pratico
        -gameBoard presenta un metodo (setGame) che dovrebbe essere implementato in Game
        -in PlayerBoard alcuni attributi e metodi sono ridondanti e dovrebbero essere ottimizzati 

\section{Confronto tra le architetture}

Confrontando i nostri due elaborati potremmo prendere spunto dal loro e utilizzare un array di interi invece di singole variabili per contare gli studenti di ogni colore, 
dato che come scrittura è senz'altro più sintetica.

\end{document}
