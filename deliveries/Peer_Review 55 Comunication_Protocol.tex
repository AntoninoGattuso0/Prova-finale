\documentclass[12pt]{article}
\usepackage[utf8]{inputenc}
\usepackage[T1]{fontenc}
\usepackage[italian]{babel}

\title{Peer-Review 55: Protocollo di Comunicazione}
\author{Gambarin Paolo, Gattuso Antonino, Mersam Rebeca\\Gruppo 56}

\begin{document}

\maketitle

Valutazione della documentazione del protocollo di comunicazione del gruppo 55.

\section{Lati positivi}

Il protocollo è molto dettagliato nel tipo di messaggio e mossa, in quanto fornisce in modo specifico
argomenti e possibili risposte. Ogni messaggio scambiato tra Client e Server è molto semplice e
intuitivo, anche chi non conosce la struttura del progetto riesce a capire qual è l’intenzione dei
progettisti.
La struttura del protocollo di comunicazione è molto orientata alla scrittura della CLI e GUI.

\section{Lati negativi}

Nel protocollo di comunicazione sono presenti solo i sequence diagram di Ping e Login, mancano i
restanti.
È abbastanza confusionaria la suddivisone delle fasi, perché i messaggi di errore sono separati da
quelli a cui si riferiscono; inoltre, in caso di errore non vengono forniti i successivi messaggi per
reinserimento della scelta (carta assistente, nuvola, pedina, etc.) o altre azioni.
In “Ping” abbiamo osservato una piccola differenza tra sequence diagram e descrizione: i messaggi
di ping vengono inviati ogni 10s (secondo la descrizione), mentre nel sequence diagram ogni 15s.
In “TurnStart” l’argomento TypePhase è poco specifico in quanto divide le azioni in “PlanningPhase”
o “ActionPhase”, senza specificare il tipo di ActionPhase: movimento studenti, movimento Madre
Natura e scelta nuvola.

\section{Confronto}

Confrontando i due protocolli ci siamo resi conto che la loro logica è molto simile, tuttavia i nostri
messaggi sono in formato JSON che possono essere meno intuitivi rispetto a quelli dei nostri
compagni. Potremmo integrare i nostri messaggi con descrizioni più dettagliate.
Consigliamo, invece, al gruppo 55 di inserire i sequence diagram mancanti, per maggiore chiarezza
di scambio messaggi tra mittente e destinatario.

\end{document}
